% Preamble. Don't worry about it.
\documentclass{article}
\usepackage{setspace}
\usepackage[utf8]{inputenc}
\onehalfspacing

\begin{document}

% --- TITLE PAGE ---
\title{Donnervögel Consulting \\ Streamlined Grading System}
\author{Markus Balaski \\ Graeme Smith \\ Jordan Toering \\ Stephen Laboucane \\ Ian Pun \\ Colin Woodbury \\ Chazz Young}
\date{\today}
\maketitle
\clearpage
% ------------------

% --- REVISION HISTORY ---
% Add more entries here as the Spec undergoes major changes.
% More recent entries come first.
\textbf{Revision History}
\begin{center}
  \begin{tabular}{| c | c | c | l |}
    \hline
    Version & Date & Members & Changes\\
    \hline
    & & & Initial Functional Requirements complete.\\
    1.0 & 2014 Jan 28 (Tues) & All Members & Initial Non-functional Requirements complete.\\
    & & & Initial UI Descriptions complete.\\
    \hline
  \end{tabular}
\end{center}
\clearpage
% ------------------------

% --- TABLE OF CONTENTS ---
\tableofcontents
\clearpage
% -------------------------

% Use this as a template. Section numbers, etc, is all handled automatically.
\part{Functional Requirements}
\section{Role - The System Administrator \label{SysAdmin}}
The System Administrator is responsible for:
\subsection{Account Management}
\subsubsection{\label{Account Creation} Account Creation}
The following fields are necessary:
\begin{itemize}
  \item The account owner's name
  \item A temporary password for the account owner
  \item The account type
\end{itemize}
\subsubsection{Existing Account Modification}
The System Admin can do:
\begin{itemize}
  \item Blocking or permission changing for accounts that are security risks.
  \item Password resets for users with forgotten passwords/multiple bad logins.
\end{itemize}
\subsection{System Log Management}
\subsection{System Backups/Restorations}
\subsection{Application Installation}

\section{Role - The Academic Administrator \label{AcAdmin}}
\subsection{The Privileges of the Academic Admin}
The Academic Admin does not have any jurisdiction over the role of System Administrator.
The Academic Administrator role assumes all functions of and can change/update any material
provided by the following users:
\begin{itemize}
  \item Assistant Academic Administrator in section \ref{AsAcAdmin} on page \pageref{AsAcAdmin}.
  \item Instructors  in section \ref{Instructor} on page \pageref{Instructor}.
  \item TA/TMs in section \ref{Marker} on page \pageref{Marker}.
\end{itemize}

\section{Role - The Assistant Academic Administrator \label{AsAcAdmin}}
An Assistant Academic Administrator is able to create, modify, copy, and delete courses.
\subsection {Course management}
\subsubsection{Creating a Course \label{courseCreation}}
To create a course, an Admin. Assistant must provide the following:
\begin{itemize}
	\item Name of course
	\item Course ID
	\item Assigned Teacher Assistants/Markers names
	\item Assigned Instructor name
	\item Start and end date of course
	\item List of activities in the course
	\item List of students enrolled in the course
\end{itemize}
\subsubsection{Modifying a Course}
An Admin. Assistant can to make any changes or additions to existing courses
freely after it has been created except for the rubric of an activity.
\subsubsection{Updating Student Lists}
An Admin. Assistant can fetch the latest list of students registered in a
specific course through the Course Management System.
When inputting the list of the students into the course, the course management
system will check for any updates from the previous list with the new one,
making any necessary additions or removal of students.
\subsubsection{Copying a Course}
An Admin. Assistant is allowed to copy a course. Copying a course will transcribe a new copy of these existing specifications:
\begin{itemize}
  \item Name of course
  \item Course ID
  \item List of activities including their individual:
    \begin{itemize}
    \item Description
    \item Marking rubric
    \item Language of the activity
    \item Type of activity
    \item Marking scheme (group/individual based)
    \item Due date
    \item Solution
    \item Test input and output files (for programming activities only)
    \end {itemize}
\end {itemize}
\subsubsection{Modifying a Rubric}
Modifying the rubric of an activity after it has been marked requires the whole
activity to be re-marked.
\subsubsection{Deleting a course}
An admin assistant is able to delete any existing course.

\section{Role - The Instructor \label{Instructor}}
The instructor has the main control over the contents of a course, its students, 
and its TAs or TMs. They can also grade any activities in the course.
\subsection{Activity Management \label{ActivityManagement}}
Activity details include:
\begin {itemize}
	\item Name of activity
	\item Activity description
	\item Rubric of activity
	\item Type of activity
	\item Language of activity (programming or actual language)
	\item Group or individual activity
	\item Due date
	\item List of student grades
	\item Solution
	\item Test input and output files (for coding activities only)
\end {itemize}
\subsubsection{Create an activity \label{CreateAct}}
Create an activity (assignment) and include the necessary details about it such 
as rubric and description.
\subsubsection{Modify an activity \label{ModifyAct}}
Edit any existing part of an activity, including its rubric. If the rubric is updated
the activity must be regraded fully.
\subsubsection{Copy activities}
Copy activities or a list of activities from a previous offering of the course, excluding
course offering specific info, but including details of the activity and its rubric.
\subsubsection{Enter tests/comparisons for code assignments}
Create tests or solutions for coding assignments that the students' work can be compared
to in grading.
\subsection{Student Management}
\subsubsection{Create student groups}
The instructor can create groups of students for an individual activity, or
multiple activities. 
\subsection{TA/TM Management}
\subsubsection{Assign TA/TMs to students \label{AssignTA}}
The instructor can assign TAs and TMs to subsets of students or subsets of 
groups in the class for a particular activity or set of activities. The TA/TMs will then
only be able to grade those subsets for the activities the instructor chooses.
\subsection{Grade Assignments \label{grading}}
\subsubsection{View Assignments and Rubric}
A particular student's or group's assignment can be selected so the rubric is
shown side by side during grading.
\subsubsection{Enter Grades}
Markers are able to enter grades directly into the rubric while they are viewing
the assignment and rubric side by side.
\subsubsection{Add Comments to Student Work}
Markers are able to add comments directly into the pdf if it is a pdf submission,
or directly into the code if it is a code submission.
\subsubsection{Test Code Assignments}
The system will compile and run code assignments through the submitted tests,
then present the outputs side by side with the sample correct outputs.

\section{Role - The TA/TM Markers \label{Marker}}
\subsection{Grade Assignments}
The same as the Grade Assignments section under the Instructor.
A TA/TM has can only access the assignments of the groups or students they
have been assigned to.

\section{The Login System}
\subsection{Failed Logins}
Users are granted five(5) attempts to log in successfully.  If they incorrectly enter their password all five times, their account is locked and they must reset their password with the system administrator before they can login again.
\subsection{Password Resets}
If a user wants to reset their own password(without the help of the System Administrator) they may ask the system to reset their password.  The system will send a randomly generated password to their email that they can use to login until they choose to change their own password.

\section{The Database}
\subsection{Content}
The database is responsible for storing:
\begin{itemize}
\item User Accounts
\item Courses
\item Assignments
\item Rubrics
\item Grades
\end{itemize}

\section{Logs}
There are two distinct types of logs: Academic Logs and System Logs.
\subsection{Academic Logs}
The academic log contains all records of additions and changes to the academic records.
The academic log documents grade and assignment modifications.  The academic log
is visible to every user excluding the Ta/Tm marker.
\subsection{System Logs}
The system log contains a record of the systems tasks.  It documents user sign-ins
and system backup times. The system log is only visible to the System Administrator.

% --------------------------------------------
\part{Prioritization of Function Requirements}
\section{Priority Lists}
\subsection{Core Features}
\begin{itemize}
  \item Account Creation
  \item Existing Account Modification
  \item Creating a Course
  \item Modifying a Course
  \item Modifying a Rubric
  \item Create an Activity
  \item Modify an Activity
  \item View Assignments and Rubric
  \item The Login System
\end{itemize}

\subsection{Most Important Features}
\begin{itemize}
  \item Copying a Course
  \item Copy Activities
  \item Enter tests/comparisons for code assignments
  \item Enter Grades
  \item Add Comments to Student Work
  \item Grade Assignments
\end{itemize}

\subsection{Less Important Features}
\begin{itemize}
  \item The Power of the Academic Administrator
  \item System Backups/Restorations
  \item Create Student Groups
  \item Assign TA/TM's to students
  \item Test Code Assignments
  \item The Database
\end{itemize}

\subsection{Desired Features}
\begin{itemize}
  \item Application Installation
  \item System Log Management
  \item Academic Logs
  \item System Logs
\end{itemize}

% --------------------------------
\part{Non-functional Requirements}
\section{Quality}
\subsection{Usability}
The application should be usable to all relevant university staff
regardless of computer experience.
\subsection{Accessibility}
\subsubsection{Valid Users}
Only users who have created an account with the System Administrator will be able
to use the application. Users will not be able to log in with their normal
university username and password.
\subsubsection{Availability Hours}
The system should be accessible at any time of day, except during backup times.
\subsection{Performance}
\subsubsection{User Base}
The client has specified that there will be around 200 user accounts. Ensure
support of up to \textbf{400 users.}
\subsubsection{Maximum Load}
The client has specified that up to 20 users might use the system during
peak times, thus the system should aim for no slow-down for up to \textbf{40 users.}
\subsection{Maintenance}
\subsubsection{Application Life Span}
The client has not indicated how long they plan to use the system once developed.
\subsubsection{Addition of Features}
The client has not requested a desire for the addition of new
features post release.

\section{Constraints}
\subsection{Platform}
The application will run on the Windows desktop.
\subsection{Implementation}
\subsubsection{Language}
The application will be written in Java using the Swing library for its GUI.
\subsubsection{Development Environment}
Eclipse and Emacs will be used for code creation.\\
?? will be used for document editing.
\subsubsection{Version Control}
The source code will be managed via a software version control system.
\subsection{External Resources}
\subsubsection{SQL Database}
The application will store user and grading data on an MS SQL server located on
the university campus, and thus the user's computer must have a network connection.
\subsection{Licensing}
\subsubsection{Closed Source}
The application will be closed source under the Donnervögel Draconian License.
\subsubsection{GPL Incompatibility}
No libraries with GPL licensing will be included in the system.

\section{Other Requirements}
\subsection{System Administration}
\subsubsection{Account Management}
User accounts may only be created through the System Administrator.
\subsubsection{System Backups}
Database backups will be performed by the System Administrator and not handled
through the application.

% -------------------------------
\part{User Interface Description}

\section{General}
\subsection{Layout}
The application will be laid out in pages for each of the multiple tasks that
can be performed.
\subsection{Navigation}
Any page of the application will have a means to go back to the previous
page visited.

\section{Login}
\subsection{Login Screen}
\subsubsection{Inputs}
The login screen will contain inputs for "username" and "password", along with a
button to "enter" (confirm the username and password entered) and a button to handle
forgotten passwords. The username and password will be checked against the login database,
and the forgotten password will invoke some process to reset passwords.
\subsubsection{Outputs}
Upon hitting the "enter" button, the user will either be granted access or denied based
on an incorrect password. If a user is granted access, they will be sent
to a main UI screen offering their account's options in the system. If they enter a wrong
password, they will be told so, and given 5 opportunities to enter the correct password
before their account is locked. If the "forgot password" prompt is hit, the user will be sent
to a separate interface for resetting passwords
\subsection{Description}
The login will be the first screen shown upon opening the program. The login screen 
satisfies the separation of the functional requirements by role, as well as the capacity for users
to reset their passwords.

\section{Activity Marking}
\subsection{Inputs}
\subsubsection{Grade Inputs}
Beside each entry will be an
area for the Marker to add a grade for that part of the rubric.
\subsubsection{Comment Insertion}
The user can insert comments directly into student work.
\subsubsection{Return Assignment Button}
There will be a button available to deliver a commented copy of the working
assignment back to the student.
\subsection{Outputs}
\subsubsection{Rubric Display}
All entries for the rubric will be shown.
Total grade will be calculated based on the above entries and shown at the
bottom.
\subsubsection{Student Work Display}
The student work will be visible beside the rubric and grade entry areas.
\subsection{Description}
This screen allows users with Marking privileges to grade student activities
as described in section \ref{grading} on page \pageref{grading}.\\
The user navigates here from \textbf{Course Choice} then \textbf{Assignment Choice}.

\section{System Admin Maintenance Panels}
The System Administrator's screens are exclusive to the System Admin
\subsection{Initial Screen}
The user is presented with the following options: 
\emph{Create Account}, \emph{Modify Account}, \emph{View/Manage System Log}, \emph{Manage System Backups}, \emph{Application Installation}
\subsubsection{Create Account}
This page asks for the fields described in \ref{Account Creation} \\
This page also has a button to "confirm creation" button, as well as a "return" button
\subsubsection{Modify Account}
This page is similar to the Create Account page, with the above mentioned fields filled in. \\
The buttons are also similar. There is a "confirm changes" button and a "return" button.
\subsubsection{View/Manage System Log}
This page opens the system log. 
\subsubsection{Manage System Backup}
This page has two buttons, buttons, a "back up now" button, and a "restore" button. \\
The "restore" button will prompt the user for the version s/he wished to restore to.
\subsubsection{Account Installation}
The user will be provided with a .exe file to be used to isntall the application.

\section{Course Creation/Modification}
There will be 4 buttons available, \emph{Add a course}, \emph{Edit existing course},
\emph{Copy a course}, and \emph{Delete a course}.

\subsection{Adding a course}
There will be fields open to the user to input the neccessary information described
in section \ref{courseCreation} on page \pageref{courseCreation}.

\subsection{Editing an existing course}
Ther will be a list of existing courses for you to choose from. Once the user has
picked the course to edit, there will be fields open to the user to make any
changes necessary. This satisfies req.
(SECTION HERE): editing existing course.

\subsection{Deleting an existing course}
There will be a list of existing courses for you to choose from with checkboxes
to satisfy req. (SECTION HERE) deleting a course.

\section{Activity Creation/Modification}
\subsection{Inputs}
The fields that will be shown for data entry or modification are all those listed
under \ref{ActivityManagement} Assignment Management on page \pageref{ActivityManagement} except for a list of student grades.
When an option would not be applicable for the activity, such as test input and output files
for non-programming activities, the field will not be shown.
\subsection{Outputs}
Activities being edited will have the current information shown in the appropriate field.
Activities being created from scratch will have each field empty.
When an option would not be applicable for the activity, such as test input and output files
for non-programming activities, the field will not be shown.
\subsection{Description}
Both activity creation and modification share virtually the same screen.
This screen satisfies the requirements \ref{CreateAct} and \ref{ModifyAct} for an Instructor to create and modify activities.

\section{Test Suite}
\subsection{Test Suite Screen}
\subsubsection{Inputs}
The only inputs the test suite would take would be comments to place in the submitted code as well as the user input needed to interact with the console output of the submitted code.
\subsubsection{Outputs}
The test suite would output to the screen the reference code, the submitted code and the console messages.
\subsection{Description}
The test suite screen contains both the reference code and the student submitted code in a side by side format.  It includes a section for the marker to view and interact with the output of the submitted code.  The test suite also allows markers to add comments directly to the submitted code.  The test suite is a side-by-side code marking system, as well as marker interaction with the submitted code which satisfies the functional requirement described in section \ref{grading} on page \pageref{grading}.

\section{Student/TA Group Management}
The TA Management Screen allows the instructor to (re)assign students to TA's.
\subsection{Inputs}
The user will be presented with three lists: all the TA's in the course and all students enrolled in the course, along with all activities.  TA's, students, and activities can be selected, along with respective Select All and Select None buttons.  There will also be a few buttons to relate TA's and students:
\begin{itemize}
  \item Assign TA
  \item Default Split
\end{itemize}

\subsection{Outputs}
\subsubsection{Assign TA}
Groups all selected students with the selected TA.
\subsubsection{Default Split}
Chunks all selected students (or all enrolled, if none are specified) by name, then assign equal sized groups to each selected TA (or all available, if none are specified).  This also fills the role of assigning all students to one TA if there is only one TA for the course.
\subsubsection{Result}
The result of the selected operation will be displayed afterwards, saying whether the process was successful or an error occurred.
\subsection{Description}
This UI page satisfies the functional requirements outlined in \ref{AssignTA} on page \pageref{AssignTA}.

\end{document}
% Nothing past this will be included in the document.
