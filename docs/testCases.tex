% Preamble. Don't worry about it.
\documentclass{article}
\usepackage{setspace,graphicx,fancyhdr}
\usepackage[utf8]{inputenc}
\usepackage[left=1in,top=1in,right=1in,bottom=1in]{geometry} % Document margins
\onehalfspacing

% For custom footers
\pagestyle{fancy}
\fancyhead{}                        % clear all header fields
\renewcommand{\headrulewidth}{0pt}  % no line in header area
\fancyfoot{}                        % clear all footer fields
\fancyfoot[LE,RO]{\thepage}         % for page #s
\fancyfoot[RE,LO]{\includegraphics{../images/logo/markshark-1x}}

% Setting the depth for Table of Contents
\setcounter{tocdepth}{2}

\begin{document}

% --- TITLE PAGE ---
\title{Donnervögel Consulting \\ MarkShark Grading System \\ Testing Document and Post Mortem}
\author{\textbf{Phase Lead: Stephen Laboucane}}
\maketitle
\centerline{\includegraphics{../images/logo/markshark-10x}}
\clearpage
% ------------------

% --- REVISION HISTORY ---
\textbf{Revision History}
\begin{center}
  \begin{tabular}{| c | c | c | l |}
    \hline
    Version & Date & Members & Changes\\
    \hline
    1.0 & 2014 Mar 28 (Fri) & Stephen L. & Document created.\\
    & & Graeme S. & \\
    & & Colin W. &\\
    & & Markus B.&\\
    & & Jordan T. &\\
	& & Chazz Y. &\\
    & & Ian P. &\\
    \hline
    1.1 & 2014 Apr 5 (Sat) & Stephen L. & Post Mortem section added.\\
    & & Graeme S. & \\
    & & Colin W. &\\
    & & Markus B.&\\
    & & Jordan T. &\\
    & & Chazz Y. &\\
    & & Ian P. &\\
    \hline
  \end{tabular}
\end{center}
\clearpage
% ------------------------

% --- TABLE OF CONTENTS ---
\tableofcontents
\clearpage
% -------------------------

\section{White Box} 
\subsection{Purpose}
% Describe the purpose of the method
\textit{loginButtonActionPerformed( )} in the LoginScreen class is called when the user clicks the login button on the login screen.  It takes the form data, and checks whether it is valid.  If true, then display the proper home page for the current user role.

\subsection{Code}
% Analyze the code for the method and break it into appropriate numbered blocks
\begin{verbatim}
private void loginButtonActionPerformed(ActionEvent evt) {
===============================================================================
1
    String name = username_field.getText();
    String pass = new String(pass_field.getPassword());
    Account test;
===============================================================================
2
    if(pass.length() < 0) 
===============================================================================
3
    {  // Temporary! Should be 5 or so.
        JOptionPane.showMessageDialog(this, "The password given is too short.");
    }
===============================================================================
4
    else if(name.isEmpty())
===============================================================================
5
    {
        JOptionPane.showMessageDialog(this, "No user name given.");
    }
===============================================================================
6
    else if(test == null) 
===============================================================================
7
    {
    	JOptionPane.showMessageDialog(this, "Invalid username/password combo");
    }
===============================================================================
8
    else
    {
        System.out.println("Logging in as " + name +
    		       " with password `" + pass + "`");
        // TODO: This needs to be passed a legit Account object.
        master = new MasterFrame(test);
        this.setVisible(false);
        master.run();
    }
===============================================================================
}
\end{verbatim}

\subsection{Flow Chart}
% Develop a flow chart for your method based on the blocks of code the method was broken into in the previous step
\centerline{\includegraphics[scale=1.1]{../images/Whitebox_Testing_Diagram}}

\subsection{Coverage}
% Comment on the types of coverage needed to test the method
Statement coverage is sufficient, because all blocks have exactly one input branch, except for Exit.  All blocks one level above Exit have only one output branch (leading to Exit) so all paths through those blocks cover all paths to Exit.  In this way, branch testing is redundant.  Additionally, there are no loops, so a block can be executed at most once in a case.  As all branches have been covered, this would be unnecessary as well.

\subsection{List of Test Cases}
% Develop a list of test cases that will adequately test the method. Do not fully develop each test case. ONLY list the coverage of each test and explain why the test is needed/useful
\begin{center}
  \begin{tabular}{| c | l |}
    \hline
    ID & Coverage\\
    \hline
    1S & START-1-2-3-EXIT\\
    2S & START-1-2-4-5-EXIT\\
    3S & START-1-2-4-6-7-EXIT\\
    4S & START-1-2-4-6-8-EXIT\\
    \hline
  \end{tabular}
\end{center}

\subsection{Detailed Test Case}
% Choose one of the test cases and describe it in detail in terms of the test case format used in class \\
% Demonstrate how your selected test case (input values) will help test the intended coverage.
\subsubsection{Test ID}
3S

\subsubsection{Test Purpose}
Unit test method \textit{loginButtonActionPerformed( )} of the LoginScreen class using White Box testing strategy (statement coverage: START-1-2-4-6-7-EXIT)

\subsubsection{Requirement Number}
\#6


\subsubsection{Testing Procedure}
Enter a username and password both of length > 5, "username" and "password".  Click the Log in button.

\subsubsection{Evaluation}
% How to evaulate it.  I don't think anything happens.
Login denied.

\subsubsection{Expected Behaviours and Results}
An error message should pop up saying "Invalid username/password combo," the user should still be on the login page, and therefore not be logged into the system.

\subsubsection{Actual Behaviors and Results}
[Added at test time]

\section{Black Box}
\subsection{Purpose}
% Describe the purpose of the method, its parameters (arguments), and any preconditions necessary for successful execution of the method
\textit{execQuery(String) : ResultSet} executes an SQL query provided as the String argument, and returns the ResultSet storing the query's response.  The database connection details are defined in another function; they currently require the program to be run on the the Server's local network.  The database also must be operational.

\subsection{Equivalence Classes}
% Describe the equivalence classes for each of the inputs and preconditions of this method

\centerline{\includegraphics[scale=1]{../images/EquivalenceClass_LAN}}
\centerline{\includegraphics[scale=1]{../images/EquivalenceClass_Server}}
\centerline{\includegraphics[scale=1]{../images/EquivalenceClass_SQL}}

\subsection{Choosing Inputs}
% Describe how you would choose representative input values to test this method
Input values would come from valid and invalid equivalence classes.  Many of these are binary, so they would not have a boundary value to input.  Should test breaking preconditions with values from all valid equivalence classes, along with a set of all invalid equivalence class values, to make sure the preconditions are correct.

\subsection{List of Test Cases}
% Briefly describe how you would use the representative values to construct a set of test cases to test the method. DO NOT develop these test cases beyond describing them in terms of the values of the parameters and the preconditions.
\begin{center}
  \begin{tabular}{| c | l | l | l |}
    \hline
    ID & On Local Network & Server Status & SQL\\
    \hline
    1 & No & Not running & Valid\\
    2 & No & Running & Valid\\
	3 & Yes & Not running & Valid\\
    4 & Yes & Running & Invalid\\
    5 & Yes & Running & Valid\\
    6 & Yes & Running & Read permission denied\\
    7 & Yes & Running & Write permission denied\\
    8 & Yes & Running & Table does not exist\\
    9 & Yes & Running & Incorrect key\\
    10 & Yes & Running & Record does not exist\\
    11 & Yes & Running & Null\\
    12 & Yes & Running & Returns 0 tuples\\
    13 & Yes & Running & Returns 1 tuple of degree 1\\
    14 & Yes & Running & Returns 1 tuple of degree more than 1\\
    15 & Yes & Running & Returns many tuples of degree 1\\
    16 & Yes & Running & Returns many tuples of degree more than 1\\
    \hline
  \end{tabular}
\end{center}

\subsection{Detailed Test Case}
% Select a single test case from those discussed in the previous item\\
% Describe your test case using the test case format seen in class.
\subsubsection{Test ID}
14

\subsubsection{Test Purpose}
Unit test CourseAccess class method \textit{execQuery( )} using Black box testing strategy

\subsubsection{Requirement Number}
\# 11.3.1


\subsubsection{Inputs}
\begin{itemize}
\item System is on the database's network
\item Database is operational
\item String = "SELECT * FROM c275g01A.dbo.Account WHERE Username = 'jtoering'"
\end{itemize}

\subsubsection{Testing Procedure}
Call \textit{execQuery}("SELECT * FROM c275g01A.dbo.Account WHERE Username = 'jToering'")

\subsubsection{Evaluation}
\begin{verbatim}
private static void testExecQueryWithAccount( ){
	ResultSet result = execQuery("SELECT * FROM c275g01A.dbo.Account WHERE Username = 'jtoering'");
	try {
    	result.next();
	    System.out.println(result.getNString(1));
    	System.out.println(result.getNString(2));
	    System.out.println(result.getInt(3));
	    System.out.println(result.getNString(4));
	    System.out.println(int type=result.getInt(5));
	    } catch (SQLException ex) {
  	        System.out.println("SQL Exception occured, the state : "
	   	    + ex.getSQLState() + "\nMessage: "
	   	    + ex.getMessage());
        }
}
\end{verbatim}

\subsubsection{Expected Behaviours and Results}
\begin{verbatim}
jtoering
61843
99999
Jordan Toering
4
\end{verbatim}

\subsubsection{Actual Behaviours and Results}
[Added at test time]

\section{Post Mortem}
\subsection{Project and Time Management}
We have learned much in regards to time estimation.  Productivity appeared to increase exponentially as deadlines neared.  A result of this is giving a small lull between initiation of a phase and the activities associated with short-term planning and "crunch time."  Time allotted to the final merging process has increased with each deliverable for two main reasons: increased complexity and overlap of pull requests, and continuing repository errors (including server downtime).
% Elaborate on "complexity of pull requests"?
For the most part, our time estimates have been accurate to within the hour.  One or two deliverables were granted a few days extension (not pushing back the final deadline, though) that could have been theoretically completed on the original schedule.  The longest phase was definitely writing the program code.  It was shorter than what we would have planned for, not being accustomed to much of the work being scheduled prior to that stage.  While the two weeks flew by, and we worked hard. Still, there was not enough time to do all we wanted.  This is both surprising and not, as we had been lead to believe that the scheduled time-line was sufficient.

\subsection{Testing}
We have yet to discover a bug in our system that has not been previously caught by the coding process itself.  Developing, debugging, and testing well thought-out automated testing systems can be fairly costly.  Some of those resources were better used in implementing a more diversified feature set.  While formal testing processes are important, MarkShark is fairly basic in its small set of inter-communications.  Good planning in class packaging was instrumental to easing this perspective of testing.  Discovering problems by trying to use methods in the program could probably  be classified a natural black box testing environment, as there is little need to dig into the implementation unless the expected output doesn't match the actual result.  Analytic white box testing should have been dealt with in some fashion while trying to get the code correct the first time.  Having different people look over and use the methods helps as well, having a fresh set of eyes to potentially catch errors.  

\subsection{Team Setting}
Working in a moderately sized group on a project of this scale has been an eye-opening experience for many of us.  A few associates had only previously worked on individual projects that could be completed in one day.
% \subsubsection{Communication}
Communication fluctuated throughout this project while testing out different methods.  Skype was the main hub, with several informal group meetings per week to facilitate face-time, in addition to a formal meeting.  Once coding began, a feature of GitHub - Issues - became useful, highlighting tasks that people were in the process of completing, items they needed to continue, or what still needed to be done.  As these are more accessible than meeting minutes, Issues could have been helpful earlier in the development cycle to track task assignment. \\
One unfortunate consequence of working in a team is the uneven ability to contribute
based on different background/former experience. Repository management fell onto those
with git experience. Writing and maintaining the database code fell onto one team
member who was the most familiar with it and the most driven. GUI creation fell to
those with the most initiative. A few of our team members were missing during key
points in the implementation phase, making them completely unfamiliar with any
section of our rapidly growing code base. This hindered the ability to contribute
evenly even further. Some of us literally put our lives on hold for two weeks for
our software, and in the end we created something we can be proud of.

\end{document}
