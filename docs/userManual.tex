% Preamble. Don't worry about it.
\documentclass{article}
\usepackage{setspace,graphicx}
\usepackage[utf8]{inputenc}
\usepackage[left=1in,top=1in,right=1in,bottom=1in]{geometry} % Document margins
\onehalfspacing

% Setting the depth for Table of Contents
\setcounter{tocdepth}{1}

\begin{document}

% --- TITLE PAGE ---
\title{Donnervögel Consulting \\ Streamlined Grading System}
\author{\textbf{Phase Lead: Ian Pun} \\ Markus Balaski \\ Stephen Laboucane \\
  Graeme Smith \\ Jordan Toering \\  Colin Woodbury \\ Chazz Young}
\date{\today}
\maketitle
\clearpage
% ------------------

% --- REVISION HISTORY ---
\textbf{Revision History}
\begin{center}
  \begin{tabular}{| c | c | c | l |}
    \hline
    Version & Date & Members & Changes\\
    \hline
    1.0 & 2014 Mar 07 (Fri) & Markus B. & Document created\\
    & & Graeme S. & \\
    & & Jordan T. & \\
    & & Stephen L. & \\
    & & Ian P. & \\
    & & Colin W. & \\
    & & Chazz Y. & \\
    \hline
  \end{tabular}
\end{center}
\clearpage
% ------------------------

% --- TABLE OF CONTENTS ---
\tableofcontents
\clearpage
% -------------------------

% ---
\section{Product Overview}  % 5 Marks
% THIS SHOULD BE 1/2 TO 1 PAGE IN LENGTH!
% Markus will do this.

% ---
\section{Getting Started}
\emph{No contents here.}

\subsection{Software Requirements}

\subsection{Hardware Requirements}

\subsection{Installation}

\subsection{Running the Application}

\subsection{UAT Installation}

% ---
\section{Functions and User Interface Description}  % 30 Marks total
% This is a very big section.
% 1. How is a function selected?
% 2. How is input entered? (Include pictures)
% 3. What keys need to be pressed?
% 4. What will the user see when a step is performed correctly?
% 5. What will the user see when a step is performed incorrectly? (errors)
\subsection{Login}

\subsection{Landing Page}
\includegraphics[scale=0.55]{../images/UIMockups/PNG_Renders/LandingPage}
The Landing Page is shown to all users of the system following login.  The Tasks
box shows actions that each user can do.  It does not show actions the user
cannot do. The above represents what an Instructor would see upon logging in.

\subsection{Manage Courses}

\subsection{Modify Courses}

\subsection{Create/Modify Activity(Detailed)}

\subsection{Copy Activity}

\subsection{Modify Rubric}

\subsection{Activity Marking(Detailed)}
\begin{enumerate}
  \item Log in to the system as a user who has marking privileges.
  \item The following screen will be shown.  Press the \textbf{Marking} button.
  \begin{center} 
   \includegraphics[scale=0.55]{../images/UIMockups/PNG_Renders/LandingPage}
  \end{center}
  \item The following screen will be shown.  Select the course that you wish
    to do marking for from the drop-down list, then press \textbf{Ok}.
    \begin{center} 
      STEPHEN'S COURSE SELECTION IMAGE HERE
    \end{center}
  \item The following screen will be shown.  Students are listed down the left
    side. Available activities are listed along the top.  Find the student you
    wish to mark for, and select the corresponding activity, then press \textbf{Ok}.
  \begin{center} 
    STEPHEN'S STUDENT/ACTIVITY MATRIX IMAGE HERE
  \end{center}
  \item The assignment's marking screen will be shown.  The marking screen
    displays the rubric, sample solution, and the student's submission.
  \begin{center} 
   \includegraphics[scale=0.55]{../images/UIMockups/PNG_Renders/activityMarking}
  \end{center}
    Perform the grading process as follows:
    \begin{enumerate}
      \item Perform necessary analysis on the student's work. 
      \item Read rubric points and enter a number into the available box 
        based on your analysis of the student's work.
      \item Repeat step (a) for remaining rubric points.
      \item A total will be shown at the bottom of the rubric to reflect the
        student's final grade on the activity.
       \item Click the \textbf{Submit} button to update the marks database with
         the changes made.
    \end{enumerate}

  \item If desired: Press the \textbf{Submit} button to move to the next
    student's submission of the same activity.
\end{enumerate}
\textbf{*NOTE: }If the user clicks \textbf{Back}, \textbf{Next}, \textbf{Logout}
or \textbf{Anywhere in the Breadcrumb} when there are unsubmitted changes a
prompt will be shown asking the user to confirm that they intend to navigate away
from the current page without saving. (As shown below)

\begin{center}
  \includegraphics[scale=0.55]{../images/UIMockups/PNG_Renders/activityMarkingWarning}
\end{center}

\subsection{Test Suite}
\begin{center}
\includegraphics[scale=0.55]{../images/UIMockups/PNG_Renders/SRS_TestSuite_Split}
\end{center}
The Test Suite shows the \textbf{Solution Output}, the \textbf{Student Output},
and the \textbf{Diff}.
The three windows can be docked together(tabbed) or positioned that each
window takes up 1/3 of the screen space.
The \textbf{Diff} window intelligently shows the difference between
solution and submitted code.

\subsection{Manage Database}
+%INSERT DATABASE MANAGEMENT PAGE HERE%
The user will be led to a page with two options: \emph{Backup Database} and
\emph{Restore Database}
\begin{itemize}
  \item Backup Database
    By clicking this, the system will create a backup of the entire system.
    It will be suffixed with the data of backup. A confirmation message will
    appear when completed.
  \item Restore Database
By clicking this, the user will be given a popup to select the backup to restore the database to (organized from newest to oldest). It will then prompt a second time to confirm the restoration of hte system to that backup.

%INSERT SELECT BACKUP HERE%
\end{itemize}
\subsection{Manage User Accounts}
%INSERT ACCOUNT MANAGEMENT FORM HERE%
A form will appear on clicking this. 
If the user wishes to modify an existing account: \\
the user must check the "Modify" check-box. The grayed out drop down menu and
save button will become active, and the "Create" button will be grayed out.
The user will select the account to edit (organized alphabetically by username).
Upon selecting a user, the form fields will populate and become editable. After
the user has finished editing the fields, the user must click "save". The system
will save the changes and return to the Landing page. \\
\emph{*NOTE*} If the user wishes to generate a new password, a confirmation
prompt will appear asking for confirmation \\
If the user wishes to create a new account: \\
The user must not check the "Modify" box. By doing so, the "Save" button and
the drop-down menu for selecting an account will be grayed out. The user then
fills the form as normal. After the user has finished,
the user must click "Create".
\subsection{Manage System Logs}
%INSERT SYSTEM LOG SELECT HERE%
Upon clicking this button, the user will be prompted with a drop-down menu of
the system logs (organized chronologically, with the default being the log
from the current day). The system will then display the log activity for
that day.

% ---
\section{Quick Reference}  % 9 Marks

% ---
\section{Known Bugs}
There are no bugs, our software is perfect.

\end{document}
